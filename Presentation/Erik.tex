\section{Introduction}
\begin{frame}
	\begin{center}
		\Huge Introduction
	\end{center}
\end{frame}

\begin{frame}
\frametitle{Who Are We?}
\begin{itemize}
	\item Erik
	\item Lasse
	\item S�ren
	\item Lars
	\item Dennis
	\item Mathias
\end{itemize}
\end{frame}

\begin{frame}
\frametitle{What Will be in This Presentation?}
\begin{itemize}
	\item What are we dealing with? - Erik
	\item What problems are there to address? And what solutions did we consider? - Erik
	\item What is the Internet of Things and how does the solution fit within the context of that? - Lasse
	\item Technologies - Lasse
	\item How did we structure the solution? - S�ren
	\item A demonstration of the outcome. - Lars, Dennis
	\item What can be concluded from our work? - Mathias
\end{itemize}
\end{frame}

\section{Bicycle Sharing System}
\begin{frame}
	\begin{center}
		\Huge Bicycle Sharing System
	\end{center}
\end{frame}

\begin{frame}
\frametitle{What Is a Bicycle Sharing System?}
	\begin{center}
		\textit{"Short term bicycle rental available at unattended stations"} - Paul DeMaio, MetroBike LCC
	\end{center}
	
	The idea is that: "bikes provide a new paradigm in short distance transport that is a realistic alternative to the bus, tram, taxi, or private car [...]".
	\newline\newline
	Pick your bicycle at A and place it at B.
	\bigskip
	\btVFill
	\tiny source of definition: http://www.bicyclesharingsystems.com/whatis.php
\end{frame}

\begin{frame}
\frametitle{The Current System}
	%picture if there is room for it
	\begin{itemize}
		\item CIVITAS ARCHIMEDES.
		\item 2009: 135 bicycles and 19 stations.
		\item Today: +2 stations and +102 bicycles.
		\item Open system, free to use.
		\item Each bicycles is used minimum 1.4 times during a day.
		\item AFA JCDecaux.
	\end{itemize}
\end{frame}

\begin{frame}
\frametitle{Problems With the Current System}
	\begin{itemize}
		\item You do not know beforehand if there are bicycles available for you and how many.
		\item You cannot make sure that there will be a bicycle for you.
		\item There is no way for Aalborg Kommune to monitor the use of the bicycles.
	\end{itemize}
\end{frame}

\begin{frame}
\frametitle{How Can We Solve These Problems?}
	\begin{large}
	Does other bicycle sharing systems address these problems?
	\end{large}
	\begin{itemize}
		\item Status: All of the examined systems provides a website for getting the current amount of bicycles at each station.
		\item Booking: Of the examined systems it is done either by creating a special subscription, paying by week, month or quarter, or by
		creating a booking for a bicycle online in advance asking for a confirmation half an hour before reservation.
		\item Monitoring: Of the examined systems it is either done through registering bicycles in docks at the stations or by 
		tracking their position with GPS.
	\end{itemize}
\end{frame}

\section{Our Goal For the Project}

\begin{frame}
	\begin{center}
		\Huge Our Goal For the Project
	\end{center}
\end{frame}

\begin{frame}
\frametitle{Problem Definition}
	%We define the problem definition of enhancing the user experience by targeting some of the mentioned problems by this hypothesis:
	\begin{center}
		\textbf{It is possible to develop a system for Aalborg Bycykel that is user friendly and manageable, within the context of the Internet of Things.}
	\end{center}
	\begin{itemize}
		\item What are the requirements for a city bicycle booking and positioning system?
		\item How can the booking and positioning system be designed and implemented within the context of the Internet of Things?
		\item Why should the developed system be used over the currently used system?
	\end{itemize}
\end{frame}

\begin{frame}
\frametitle{Requirements}
	We want a system that is able to:
	\begin{itemize}
		\item Provide status of availability of bicycles.
		\item Give an option to book a bicycle.
		\item Track the position of each bicycle.
		\item Collect data for analysis of the usage for Aalborg Kommune.
		\item Predicting when a bicycle will be available for use.
	\end{itemize}
\end{frame}

\begin{frame}
\frametitle{Criteria}
	What values did we want the system to have?
	\begin{itemize}
		\item As little change as possible, rather an extension to the current system.
		\item Easy to use and accessible by everyone.
	\end{itemize}
\end{frame}

%Visualise the different suggestions to problems(may be too specific)
\section{Chosen Solutions}
\begin{frame}
	\begin{center}
		\Huge Chosen Solutions
	\end{center}
\end{frame}

\begin{comment}


\begin{frame}
\frametitle{Availability of Bicycles - Camera}
\begin{figure}
\centering
\includegraphics[scale=0.05]{camera}
\end{figure}

\textit{Small change to the existing system, but not easy to use (interpret).}

\bigskip
\btVFill
\tiny source of image: http://www.mantratec.com/Weatherproof-cameras/weatherproof-camera-1.jpg

\end{frame}

\begin{frame}
\frametitle{Availability of Bicycles - Chip}
\begin{figure}
\centering
\includegraphics[scale=0.2]{RFID-Chip}
\end{figure}

\textit{Requires an addition of a chip to every bicycle.}\\
\textit{Addition of entrance and exit gates at the stations.}\\
\textit{Easily understood by the user.}
\bigskip
\btVFill
\tiny source of image: https://pbs.twimg.com/profile\_images/1280492977/RFID-Chip.jpg
\end{frame}
\end{comment}
\begin{frame}
\frametitle{Availability of Bicycles - Dock}
\begin{figure}
\centering
\includegraphics[scale=0.4]{dock}
\end{figure}

\textit{Requires a new system for docking a bicycle.}\\
\textit{Easily understood by the user.}\\
\textit{It integrates well with a locking mechanism.}

\bigskip
\btVFill
\tiny source of image: http://www.tuvie.com/wp-content/uploads/public-bike-system1.jpg
\end{frame}
\begin{comment}

\begin{frame}
\frametitle{Unlocking of Bicycles - On time}
\begin{figure}
\centering
\includegraphics[scale=0.2]{clock}
\end{figure}

\textit{Easily understood by the user - just be there on time.}\\
\textit{Not at all flexible, hence not that user friendly.}

\bigskip
\btVFill
\tiny source of image: http://images.all-free-download.com/images/graphiclarge/purzen\_clock\_face\_clip\_art\_24896.jpg
\end{frame}

\begin{frame}
\frametitle{Unlocking of Bicycles - GPS}
\begin{figure}
\centering
\includegraphics[scale=0.1]{gps}
\end{figure}

\textit{Requires the user to carry a GPS device (not necessarily all users have such device).}\\
\textit{Can cause false unlocking.}

\bigskip
\btVFill
\tiny source of image: https://cdn3.iconfinder.com/data/icons/wireless/512/16-512.png
\end{frame}

\begin{frame}
\frametitle{Unlocking of Bicycles - QR Code}
\begin{figure}
\centering
\includegraphics[scale=0.1]{QR}
\end{figure}

\textit{Requires the user to have a device (often a smartphone) to take a picture of a QR code.}\\
\textit{May not be easily understood by all users.}\\

\bigskip
\btVFill
\tiny source of image: http://www.qrstuff.com/images/sample.png
\end{frame}

\begin{frame}
\frametitle{Unlocking of Bicycles - SMS}
\begin{figure}
\centering
\includegraphics[scale=0.1]{SMS}
\end{figure}

\textit{Requires a cellphone.}\\
\textit{Easy to use by most users.}\\
\textit{May be expensive for users without Danish subscription.}

\bigskip
\btVFill
\tiny source of image: http://www.fusionindia365.com/images/templatemo\_logo.png
\end{frame}
\end{comment}
\begin{frame}
\frametitle{Unlocking of Bicycles - Password}
\begin{figure}
\centering
\includegraphics[scale=0.3]{password}
\end{figure}

\textit{Requires hardware for entering the password.}\\
\textit{Easily understood by the user.}\\
\textit{No requirements for the user.}

\bigskip
\btVFill
\tiny source of image: http://i.bnet.com/blogs/272104\_b9383b0487-300x225.jpg
\end{frame}

\begin{comment}

\begin{frame}
\frametitle{Booking Software - Mobile App}
\begin{figure}
\centering
\includegraphics[scale=0.1]{smartphone}
\end{figure}

\textit{Requires a smartphone with internet connection.}\\
\textit{Not accessible by everyone.}

\bigskip
\btVFill
\tiny source of image: http://cdns2.freepik.com/free-photo/smartphone\_318-23380.jpg
\end{frame}
\end{comment}



\begin{frame}
\frametitle{Booking Software - Website}
\begin{figure}
\centering
\includegraphics[scale=0.1]{web}
\end{figure}

\textit{Requires a browser.}\\
\textit{Can be accessed by everyone (e.g. go to the library)}\\
\textit{Can be adapted for mobile screen resolutions and take advantage of e.g. touch screen gestures.}

\bigskip
\btVFill
\tiny source of image: http://www.eventmanagerblog.com/uploads/2014/03/creating-perfect-event-websites.png
\end{frame}


\begin{frame}
\frametitle{Tracking of Bicycles - GPS}
\begin{figure}
\centering
\includegraphics[scale=0.15]{gps-signal}
\end{figure}

\textit{Requires a GPS device attached to each bicycle.}

\bigskip
\btVFill
\tiny source of image: http://cdns2.freepik.com/free-photo/receiving-gps-signal\_318-10003.jpg
\end{frame}
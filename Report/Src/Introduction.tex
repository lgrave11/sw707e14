%disposition:
%politisk mål
	%infrastruktur(letbane, CO2, )
	%sundhed archimedesaaaa
	
%aalborg bycyklen
%problemer
	%ingen statistik og umuligt at finde om der er en cykel uden at gå hen til station
	%ingen sikring om at der er en cykel tilstede
	
%vores løsning

In the current time of Danish politics, a heavy focus has been put on better health \citep{misc:nationalemaalhelbred}.
Additionally, a great focus has been placed on climate change, and how to tackle this \citep{misc:klima}.
A part of a solution to this is getting more people to use bicycles for transportation especially in urban areas.
This limits CO2 pollution due to people not driving in cars and increase health of people due to exercising when bicycling.
A way to make people bicycle more are bicycle sharing systems \citep{misc:impactofbikeshare}, which appear in several cities \citep{misc:cibi, misc:bycyklen, misc:AltaBicycleShare, misc:aalborgbycykelMain}.

The bicycle sharing system that we focus on is Aalborg Bycykel.
It is a system where several bicycle stations are placed around the city of Aalborg, and when you need a bicycle, you travel to one of those stations and retrieve a bicycle.
Then when you are finished using the bicycle you deliver it back to one of the stations.
However, some immediate problems are associated with the currently active system.

One of the problems with the system is that bicycles can easily get lost and there is no way to locate missing bicycles, other than user reports.
Additionally, for a user to know if some bicycle is available, he has to travel from station to station until he finds an available bicycle.
Furthermore, if a user wants to be more certain that he can retrieve a bicycle in the near future, there is no way to ensure this other than retrieving a bicycle ahead of time.

These are some of the central issues that is sought to be resolved with the developed system described in the following chapters.
In the developed system, we take other existing bicycle sharing systems into account \citep{misc:cibi, misc:bycyklen, misc:AltaBicycleShare}.
On the basis of this, a booking and status system is developed for the users, and a tracking and statistics system is developed for Aalborg Kommune.

Some of the technical challenges we encounter are communication between different parts of the system, which we achieve with SOAP and TCP, how to synchronise state across these different parts, which we address with well considered rules.
Additionally, we discuss how to design the system to reach a good usability standard, where we prioritise simple solutions over advanced solutions and in addition to this utilise usability tests.

Chapter 2 documents the analysis, resulting in a problem definition as well as a specification of the requirements and target audience.
Chapter 3 documents suggested solutions, where solutions to different problems in the problem domain are suggested.
Chapter 4 documents technologies, where different technologies and concepts used for this project are discussed.
Chapter 5 documents design , where the design of the system is discussed.
Chapter 6 documents the implementation, where the implementation of the designed system is explained.
Chapter 7 documents tests, which includes Unit and usability tests. 
Chapter 8 documents the discussion of decisions about the overall system and why they were made. The developed system is compared to other existing systems. Additionally, further development options are documented.
Chapter 9 concludes the project, where the hypothesis is verified according to the questions asked in the problem definition.
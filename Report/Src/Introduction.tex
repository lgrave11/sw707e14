%disposition:
%politisk mål
	%infrastruktur(letbane, CO2, )
	%sundhed archimedesaaaa
	
%aalborg bycyklen
%problemer
	%ingen statistik og umuligt at finde om der er en cykel uden at gå hen til station
	%ingen sikring om at der er en cykel tilstede
	
%vores løsning

In the current time of Danish politics, a heavy focused has been lain upon healthiness \citep{misc:nationalemaalhelbred}.
Additionally, a great focus has been placed on climate change, and how to resolve this \citep{misc:klima}.
A part of a solution to this is people bicycling more in urban areas.
A way to make people bicycle more are bicycle sharing systems \citep{misc:impactofbikeshare}, which have appeared in several cities \citep{misc:cibi, misc:bycyklen, misc:AltaBicycleShare, misc:aalborgbycykelMain}.

One of those systems that we focus on is Aalborg Bycykel.
It is a system where several bicycle stations are placed around the city of Aalborg, and when you need a bicycle, you travel to one of those stations and pick a bicycle.
Then when you are finished using the bicycle for the day you deliver it back to one of the stations.
However, some immediate problems are associated with the current active system.

One of the problems with the system is that bicycles can easily get lost and there is no current way to locate the missing bicycles, other than user reports.
Additionally, for a user to know if some bicycle is available, he has to walk to stations until he finds an available bicycle.
Furthermore, if a user wants to be more certain that he can retrieve a bicycle in the near future, there is no way to ensure this other than retrieving a bicycle ahead of time.

These are some of the central issues that is sought to be resolved with the developed system described in the following chapters.
In the developed system, we take other existing bicycle sharing systems into account \citep{misc:cibi, misc:bycyklen, misc:AltaBicycleShare}.
On the basis of this, a booking and status system is developed for the users, and a tracking and statistics system is developed for Aalborg Kommune.
\section{Usability Test}
This section covers the usability tests performed on the system.
Tests have been performed on the public part of the website.
The administrator site was not tested because no potential users were available to do so.

These are the tasks:

\begin{enumerate}
\item Establish an overview
\item Status for bicycle
\item Booking - Login
\item Booking - Time and booking
\item Cancel planned booking
\item Examine the booking history
\end{enumerate}
For a full description of the tasks see \appref{app:usability-test}.

The usability problems were uncovered using \textbf{I}nstant \textbf{D}ata \textbf{A}nalysis (IDA) \citep{misc:usabilitytest}.

The procedure for IDA is as follows:

\begin{itemize}
\item The test monitor person introduces the test subject to the evaluation procedure. After which the test monitor gives the first task to the test subject.
\item The test subject attempts to solve the task until they feel to have accomplished it, or until the test monitor gives them a new task.
\item Test subjects talk and explain what they are doing and how they feel they understand the overall system while doing the tests.
\item The test subjects are interviewed where they talk about their experiences when finished with all the tests.
\end{itemize}

After the test, there is a meeting with the user where problems are discussed and which they felt were significant.
The result of IDA is a set of usability problems categorised into one of three categories: Cosmetic, Serious, and Critical.
We used IDA because it catches most of the usability problems in much less time than what a more rigorous approach would have caught \citep{misc:usabilitytest}.

\subsection{Test Subjects}
The tests were performed on four test subjects, whose demographics are as follows.
\begin{enumerate}
\item Gender: Female, Age: 29, Profession: Lawyer, Technological abilities: Average
\item Gender: Male, Age: 28, Profession: Lawyer, Technological abilities: Above average
\item Gender: Female, Age: 56, Profession: Office clerk, Technological abilities: Average
\item Gender: Male, Age: 58, Profession: Electrician, Technological abilities: Above average
\end{enumerate}

By average we think of a person who can construct basic text and spreadsheet documents as well as write mails and search on the web.
For above average we think of a person who has the same skills set as an average person but where they also can delve into technical details such as a settings menu.

The tests subjects are part of the target audience, see \secref{sec:ta}, however, with more time it would be beneficial to have tests subjects with below average capabilities to cover the whole target audience and what issues might occur with such persons.
\subsection{Usability Problems}
As a result of the tests, the following usability issues were uncovered.

\begin{description}[style=nextline]
	\item[{\#}1 Fields reset]
		All subjects experienced a problem with the fields resetting if they tried to book with incorrect information.
		This is for test subject 3 particularly critical, as she double booked a bicycle.
	\item[{\#}2 Error message understandability] Test subject 4 had a problem with understanding the error message if location is not selected.
	The error message is not descriptive enough, as it says "Please fill in all fields" instead of "please choose a location".
	\item[{\#}3 Difficulty finding history] Test subject 1 and 3 both experienced some problems finding the booking history button, with subject 3 having it being a bit more severe, as it took her half a minute to find the button, whereas for subject 1, it took about 15 seconds. 
	The problem for both of them being that they did not notice the button the first time they navigated to the profile part of the site.
	\item[{\#}4 Booking/Unbooking confirmation] Subject 2 became a bit confused on whether he booked/unbooked something, as no confirmation is shown to the user.
	Additionally, he requested a confirmation box, to prevent a person with clumsy fingers to on accident book/unbook a bicycle.
\end{description}

An overview of the issues that each test subject experienced can be seen in \tabref{tab:usabilityissues}.
The issues are ranked for each test subject, such that Cos means a cosmetic issue, S means a serious issue, and Crit means a critical issue.
\begin{table}[h]
	\centering
	\begin{tabular}{ll|cccc}
		\multicolumn{1}{c}{}&\multicolumn{4}{r}{Test Subject ID}&\\
		%	\cline{3-6}
		\multicolumn{2}{c|}{}&1&2&3&4\\
		\cline{2-6}
		\multirow{2}{*}{Issue Description}& {\#}1 Fields reset & S &  & Crit & S\\
		%	\cline{2-2}
		& {\#}2 Error message understandability &  &  &  & Cos\\
		%	\cline{2-2}
		& {\#}3 Difficulty finding history & Cos &  & S & \\
		%	\cline{2-2}
		& {\#}4 Booking/Unbooking confirmation &  & Cos &  & \\
		%	\cline{2-2}
	\end{tabular}
	\caption{Usability issues overview.}\label{tab:usabilityissues}
\end{table}

As can be seen, the amount of usability issues found is very low.
This may be because the website is simple and easy to use, however, it is probably also connected to the size of the document with the usability tasks and the amount of test subjects.
Four issues are found, and for each of these issues, we go through how we correct these issues.

\begin{description}[style=nextline]
	\item[{\#}1 Fields reset]
	The issue with fields resetting when you click on book/unbook can be solved in three ways.
	One is to restructure that part of the website to use AJAX.
	Another way is to save the information of the fields in sessions, in order to be loaded when the page gets reloaded. 
	The third and most convenient way is to check if the fields has been entered correctly with JavaScript, and determine whether to perform the action based on this.
	\item[{\#}2 Error message understandability]
	This issue is tied to checking if any of the fields are empty.
	Instead, a more specific error message could be given to indicate which field has a missing entry.
	\item[{\#}3 Difficulty finding history]
	In order to solve this issue, the button size could be increased.
	It could be a matter of overlooking the button and may not be an issue that will not be present when the user have located the button at least once before.
	\item[{\#}4 Booking/Unbooking confirmation]
	The booking/unbooking issue can be solved in two ways.
	One way is to present some status text when the booking/unbooking action has been performed.
	Another way is to enforce a confirmation box for the user to agree with the action initiated.
	
	We find the first way better, for the booking action, and the confirmation box better for the unbooking action.
	This is due to booking a bicycle by accident is not as severe an action as the unbooking, since if you booked something by accident, you can unbook it after and no harm is done.
	On the other hand, if you by accident unbook something, you have a smaller chance of getting a bicycle booked again, if for instance another person books the last bicycle in the meantime.
\end{description}

An effort to correct the problems was made, and these are the changes that were made: More descriptive error messages, confirmation dialog boxes added for unbooking, and the history button was made easier to locate by making the font size larger.
Additionally we also resolved the issue with fields being reset by use of the JavaScript approach.
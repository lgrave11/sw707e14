\section{Availability of Bicycles}\label{sec:availability}
One of the requirements is that the users of the system need to be able to see how many bicycles are available at a given station.
This section covers the suggested solutions for that requirement.
We look at three different availability checking solutions, which is camera, dock, and chip based

% Camera
\subsection{Camera}
A solution that is easy to implement is to use a camera such that users of the system can visually check if any bicycles are available at a given station. 

This solution is an easy way to add some kind of availability checking without investing in a larger system.

Of course this solution has its downsides, because the information is not reliable in certain situations. 
One example of a situation where a camera would be insufficient would be if it is dark at the station.
You also have to consider that the information transmitted by the camera can be deceiving, in that it could show a bicycle that does not really belong at the station and that would make the user come to the conclusion that there are bicycles at the station.
Another problem with the camera solution is that it does not provide an exact number of bicycles available but rather leaves it up to interpretation for the user.
Another possible issue would be with bicycles being at the station but not being available to use because they are booked.
Furthermore, the amount of data needed to be transmitted from the camera to the user could become a bottleneck.

% Dock
\subsection{Dock}
Another solution that requires more work to implement is a \textit{Dock} based implementation.
The main idea is that when a bicycle is delivered at a station it is placed in a dock.
A given dock with some kind of detector is then able to register if there is a bicycle placed in it.
Each station then contains several docks, and can read from each of them to provide a number for how many bicycles are located at that station. 

%good:
% exact information about how many bikes are available
This solution, unlike the \textit{Camera} solution, provides exact information about how many bicycles are available at a given station with nothing up to interpretation for the user.

%bad:
For this system to work, the bicycles would have to be placed into the docks to provide any correct information.
That is, the system is not registering bicycles misplaced outside the docks.

% Chip
\subsection{Chip}
A similar solution to the \textit{dock} solution is to use chips, such as RFID chips. 
Such a solution is similar in that it provides the same information, namely the amount of bicycles at a given station.
The solution differs, however, in the way this is registered.

One way to register bicycles is to take inspiration from a car park.
The station would have one or more entrance gateways and one or more exit gateways that are able to read the chip.
When entering the station with a bicycle, the total amount of bicycles then gets incremented, and when leaving with a bicycle it gets decremented.

Another way, depending on the distance these chips could be read from, is the possibility to deposit the bicycles at the station and the station would continually count the amount of bicycles it detects in the near-distance.

% good
These two solutions have in common that they are easy to use, as you come and pick a bicycle and then return it when you are finished.
% bad
On the downside it requires modifying all the bicycles in the fleet, furthermore, if using the car park idea, the stations can end up taking more space than the other solutions. 

\subsection{Chosen Solution}
Among the above mentioned solutions to the problem of determining the availability of bicycles, a preferred solution has to be chosen. 

The impact of the solutions is measured on the amount of hardware needed to be added to the current system in order to get a working solution. 
For example, common to all the solutions is that they require a network connection available at every station because communication with the system is necessary. 
Furthermore, each solution require different hardware and installations in order to provide the information needed. 

The solutions in order of least impact are as follows: \textit{Camera}, \textit{Chip}, and \textit{Dock}.
With the \textit{Camera} solution, only an addition of a camera at every station is required. 
The \textit{Chip} solution on the other hand requires every bicycle being equipped with a RFID chip and readers at the stations.
The solution with the most impact, the \textit{Dock} solution requires an installation of multiple docks at stations, but it does come with some benefits.
Specifically that it is believed to be easy for the user to use, and can be more easily extended than other solutions with more functionality such as forced locking.
This assessment is based on the other existing solutions such as GoBike which uses docks.

The \textit{Camera} solution is not considered further because of the possibly inconsistent interpretation by the user, possibly leading to situations where a bicycle is interpreted by a user watching the camera feed to be available, but in fact is not.

The \textit{Chip} solution might be suitable if you only consider the availability of bicycles, however, since there is a requirement for booking of bicycles this solution is not considered further as a standalone solution.
However, it is worth considering for the \textit{Dock} solution to register which bicycle is located at a given dock.

The \textit{Dock} solution provides the system with a simple way of determining the number of bicycles docked at each station.
The \textit{Dock} solution appears to be best, though it does require some hardware installation.
This solution is also very user friendly and easy to use, which is why we think it is the one that fulfils the criteria best and is thus chosen as the preferred solution for availability of bicycles.
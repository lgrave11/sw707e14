\section{Booking Software}
In order to book a bicycle the user need to interact with some interface.
This interface could be in the form of a website or an application, each with their pros and cons.

\subsection{Website}
The user could make their booking through a website where the user could be asked to create a profile.
The advantage of requiring a user to make a booking could be that the user, as well as Aalborg Kommune, would be able to see statistics about their bicycle usages.
It does, however, have the disadvantage that it is slightly complicated to make the booing, since the user is required to login, although with modern browsers that is able to store usernames and passwords it can make it less complicated.
If the user is not required to create a profile, then in combination with the ideas about unlocking the bicycles, discussed in \secref{sec:remoteLock}, the user would still need to enter some information that can be used to identify him and his booking.
It could be argued that in the long term it would be simpler to create a profile and login, rather than entering the same information for each booking.

\subsection{Application}
The application could be in the form of a mobile application. 
This would simplify the booking process for the users on the move, because the application could be optimised for easy access using touch screen gestures.
The mobile application would exclude people without a smartphone unless both a mobile application and a website are developed.
However, the mobile application could be in the form of a website optimised for phones, and thus allow people to easily use their phone, as well as their desktop computer for booking. 

\subsection{Chosen Solution}

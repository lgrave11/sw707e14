\section{Booking Software}
In order to book a bicycle the user needs to interact with an interface.
This interface could be in the form of a website or an application, each with their pros and cons.

\subsection{Website}
The users could make their booking through a website where the users could be asked to create a profile.
The advantage of requiring a user to make a booking could be that the user, as well as Aalborg Kommune, would be able to see statistics about their bicycle usages.
It does, however, have the disadvantage that it is slightly complicated to make the booking, since the user is required to login, although with modern browsers that capable of usernames and passwords, it does not necessarily have to be very complicated.
If the user is not required to create a profile, then in combination with the ideas about unlocking the bicycles, discussed in \secref{sec:Lock}, the user would still need to enter some information that can be used to identify him and his booking.
It could be argued that in the long term it would be simpler to create a profile and login, rather than entering the same uniquely identifiable information for each booking.

\subsection{Application}
The application could be in the form of a mobile application. 
This would simplify the booking process for the users on the move, because the application could be optimised for easy access using touch screen gestures.
The mobile application would exclude people without a smartphone unless both a mobile application and a website is developed.
However, the mobile application could be in the form of a website optimised for phones, and thus allow people to easily use their phone, as well as their desktop computer for booking. 

\subsection{Chosen Solution}
For this project it was decided to focus on only one application.
We decided to focus on a website, since it can be optimised for use on mobiles phones, as well as for desktop computers. 
If the solution had been a dedicated mobile application it would limit the possible users to be people with smartphones, and possibly primarily Danish users because of internet roaming prices.

We chose that the better solution is for the user to create a profile in order to book bicycles. 
This allows Aalborg Kommune to keep better track of who was booking the bicycles and the user would not need to enter their user information each time they want to make a booking.
The users who do not book a bicycle should be able to just borrow one of the bicycles that is not already booked without creating a user.
They would therefore not be needed to use the website, but should still be able to see how many free bicycles there are at a given station.
This was decided because we believe it would be too inconvenient to create a profile if you quickly want to borrow an available bicycle.
\section{Other Existing Systems}\label{sec:existing-systems}
A part of analysing and designing a city bicycle system is to investigate the current solutions on the market. 
Therefore an evaluation of existing systems is conducted.
We found several existing systems implementing different aspects and features of a city bicycle system. 
These systems are listed below:
\begin{itemize}
\item Bycyklen in Copenhagen by Gobike
\item Cibi by AFA JCDecaux
\item Alta Bicycle Share by Alta
\end{itemize}
\subsection{Bycyklen in Copenhagen by Gobike}
The bicycle share system in Copenhagen is called Bycyklen and contains 1,860 bicycles and 100 docking stations \citep{misc:bycyklen}. 
Gobike is a Danish/Dutch company that designed this smart system with a bicycle they call Smart Bike. 
The Smart Bike is equipped with an information screen in the form of a tablet providing the user with an interface to lock the bicycle, select the level of assistance from the electrical system, navigate through the city via Global Positioning System (GPS) and explore new interest points such as cafés and shops.
Furthermore, users of the bicycle system are able to check bus or train arrivals near their current location.
The Smart Bike uses the tablet to send the location, who is travelling, and other statistics about the bicycle such as battery life to Gobike Admin.
Gobike is currently looking at possibilities such as location based marketing, adjusting the traffic lights according to the cyclist based on the pattern of the bicycle trip, and the weather such as wind speed.
Bycyklen has a very simple 3-step process to rent a bicycle.
\begin{enumerate}
\item Book a Smart Bike ahead from your computer or tablet, find a Smart Bike and log in via the on-board tablet.
\item Unlock the bicycle through the tablet, drive around in the city, possibly assisted by the GPS, paying by the hour.
\item Return the Smart Bike, lock the bicycle and log out of the system via the tablet.
\end{enumerate}
The Smart Bike enables the user to lock the bicycle and securely park it during the bicycle trip.

\subsection{Cibi By AFA JCDecaux}
The Cibi bicycle system relies on SMS, where a booking of a bicycle is performed by sending an SMS to a special number with the ID of the bicycle slot in the docking station \citep{misc:cibi}. 
The user then receives an acceptance SMS and the bicycle is unlocked from the docking station. 
A bicycle trip is charged by the hour, and as with the bicycle share system in Copenhagen, a user can lock the bicycle during the trip. 
This is done through a wire lock which uses a code that the user was given in the SMS when booking the bicycle.
The system also allows for checking the amount of bicycles at a station using special docks and chips on the bicycle \citep{misc:omcibi}.

\subsection{Alta Bicycle Share}\label{subsec:alta}
Alta Bicycle Share is a company that design, deploys, and manages bicycles in USA \citep{misc:AltaBicycleShare}.
The company currently have projects ongoing in nine different cities in USA, with more than thousands of stations and over tens of thousands bicycles. 
Alta Bicycle Share believes that people get the best experience from the environment when the environment is sustainable and enjoyable.
The bicycle stations are designed such that they can be placed everywhere in the cities without any preparation, since they get electricity from solar panels.
Furthermore, the stations are using a cellular connection to upload their data to the main database, so the citizens can see if there are any bicycles at a given station.
To rent a bicycle from one of Alta Bicycle Share's system, you have to use a card, which can be bought in shops near the stations.
These cards then give access to any bicycle for a given time at any station, however, using the bicycle for a longer period\tododennis{Hvor meget?} of time can lead to a fee.

\subsection{Summary}
The Copenhagen city bicycle system, Bycyklen, includes some good features such as a booking system that works using a tablet on the bicycle that is connected to the internet.
It also includes a GPS used for navigation and tracking.
One downside to this system, however, is that the bicycles are seemingly very expensive.

Cibi is a little different in that booking happens over SMS.
At the same time it also allows for locking of the bicycle using a special code also sent over SMS.
It also allows for retrieving information about the amount of bicycles at a station using a special dock and a chip on the bicycles.

Alta on the other hand uses a card to rent bicycles, and uploads information about bicycles at stations for the users of the system to easily get an overview of where bicycles are located.

\begin{table}[H]
	\begin{tabular}{|p{0.13\textwidth}|p{0.18\textwidth}|p{0.18\textwidth}|p{0.18\textwidth}|p{0.2\textwidth}|}
		\hline                       & \pbox{20cm}{\bfseries Aalborg\\ Bycyklen}           & \pbox{20cm}{\bfseries Copenhagen\\ Gobike} & \pbox{20cm}{\bfseries Cibi}                     & \pbox{20cm}{\bfseries Alta Bicycle Share} \\ 
		\hline \textbf{Booking}               & No                            & On a tablet       & No                       & No \\ 
		\hline \textbf{Borrowing / unlocking} & Deposit 20DKK                 & Login with tablet & SMS code deposit 300DKK  & Keycard \\ 
		\hline \textbf{Tracking}              & No                            & GPS               & Dock + chip at stations  & No \\ 
		\hline \textbf{Penalty}               & Loss of deposit               & Continues payment & Loss of deposit          & Fine per half hour overtime \\ 
		\hline \textbf{Cost}                  & Deposit, but otherwise free   & First half hour free, then pay per hour    & First hour free, then pay per hour  & Depends on membership \\ 
		\hline 
	\end{tabular} 
	\caption{Comparison of bicycle systems.}
	\label{tab:bicyclecompare}
\end{table}

A comparison of the different bicycle sharing systems examined can be seen in \tabref{tab:bicyclecompare}.
As can be seen, \bycykel is lacking a few features compared to the other systems and signifies a room for improvement.
The improvements to be performed are then inspired by the other examined systems.
However, \bycykel is free to use as long as you deliver a bicycle back after use. 
This ease of accessibility of \bycykel we find valuable, and is a value we seek to keep in the developed solution.
The knowledge gained is then kept in mind for defining the problem definition and the specification of the requirements for the new system.
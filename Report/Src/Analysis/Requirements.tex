\section{Requirements}
A few system criteria have been set up that the different software and hardware solutions should conform to. 
In addition to this, software and hardware requirements are listed with the highest priority first, such that the goal for the product is to fulfil the requirements in the listed order. 
Not all requirements are necessarily fulfilled as time is limited, but we strive to fulfil as many requirements as possible.
We do not make any concrete decisions on the hardware, however, we determine some requirements for the hardware but how they are fulfilled is beyond the scope of this project.

\subsection{System Criteria}\label{sec:systemCriteria}
Below is a set of system criteria that should be considered when choosing a solution for the system.
These criteria should not be taken as requirements for the system, but as guidelines for the desired outcome of the system.
In that, when choosing a solution these criteria should be considered and discussed to help choose the solution that matches them best.

\begin{itemize}
	\item A chosen solution should not redesign the entire existing system but rather add an extension to the existing system.
	%By a lightweight extension we mean an extension that require a minimum of additions to the existing system in terms of hardware both for the bicycles and the bicycle stations.
	\item A solution should be easy to use and access by everyone. 
	It should require as few steps as possible in the process of borrowing a bicycle.
\end{itemize}

With the system criteria specified, the software requirements follows.

\subsection{Software Requirements}
These are the requirements that should be fulfilled by the software in prioritised order.

The system should be able to:
\begin{description}
\item[See how many bicycles and docks are available at a given station] \hfill \\
The users should be able to see how many bicycles and docks are available at a given station, such that the user can easily see which station to go to.
\item[Provide the option of booking a bicycle] \hfill \\
The user should be able to book a bicycle and this booked bicycle should be locked and unavailable for everyone except the one who booked it in some time frame before usage.
\item[Track bicycles] \hfill \\
The data collected by the positioning system should be used to locate the bicycles if they are not returned to a station.
\item[Obtain data for analysis] \hfill \\
The user should be able to collect information and statistics about the usage of the system.
The organization responsible should be able to use this information to, for example, perform analysis about where to redistribute bicycles and to determine if purchase of new bicycles is necessary.
\item[Predict availability of bicycles depending on placement and other variables] \hfill \\
The position should be used for predicting the usage of the bicycles as well. 
For example to predict when a bicycle is available at a given station again.
\end{description}

\subsection{Hardware Requirements}
% interface med server (fra stationerne)
%  - mulighed for at låse og låse op for en cykel
%  - vi skal være i stand til at se hvor mange cyckler der er på stationen
% interface med server (fra cyklerne)
%  - få fat i positionen af den enkelte cykel
In order to fulfil the requirements for the software listed above, a set of hardware requirements are necessary.
These requirements specify a communication interface between software and hardware, and as long as the hardware implements these interfaces, the software solution should be able to perform the required tasks.
The software requirements are listed in prioritised order, and depending on what requirements are implemented different hardware requirements need to be fulfilled. 
For example if the \textit{Track bicycles} requirement is implemented, then the hardware on the bicycles should be able to transmit its position.
\begin{itemize}
	\item Interface between station and server
	\item Interface between bicycle and server
\end{itemize}

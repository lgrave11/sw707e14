\section{The Internet of Things}
This section introduces the Internet of Things, a concept fairly central to the product developed. It also introduces various hardware considerations, though while not important to the product itself because the hardware will be simulated but it provides the context in which real-world applications for the IoT are developed.

The Internet of Things (the IoT) is the concept describing the interconnection of uniquely identifiable objects in the real world, along with their virtual representation and the interfaces that allow for manipulation and/or information retrieval regarding those objects.
This is done using various protocols in several different domains used in multiple applications \citep{misc:InternetOfThingsDefinition} \citep{misc:InternetOfThingsDefinition2} \citep{misc:InternetOfThingsDefinition3}.

The IoT manifests as subsystems to provide some public service or good, either by informing the user regarding objects in the real world or allowing an intelligent system to manage those objects.
An example of this could be a so-called 'smart home' that allows you to use a single device to manage the connected objects in your home, such as the lights or the oven in the kitchen \citep{misc:InternetOfThingsExamples}.
One society-wide use for it is the idea of a 'Smart Grid', where the electricity usage is monitored and managed by an intelligent system that will for example redirect electricity if a cable has been cut somewhere in the system\citep{misc:smartGrid}.

In the real world there are already a lot of objects on the IoT, and by 2020 it is estimated that there will be upwards of around 26-30 billion thing \citep{misc:IoTGrowth1}\citep{misc:IoTGrowth2}.
This will require at some point likely require a jump to the IPv6 protocol as the amount of IP addresses are severely limited by IPv4\citep{misc:numberOfAddresses} given that the fixes/additions to IPv4, such as multiple devices sharing a single IP address, will at some point become insufficient.

One important aspect of the objects connected to the IoT will be what technology they use to connect, here WiFi or mobile networking will not work well as they will likely interfere with one another.
The connectivity technology has to be low-power, cheap and non-interfering.
One way to do that would be to directly connect objects to a server of some sort that is connected to the internet, though that is not practical for objects that must be mobile.
RFID chips can provide relevant information to an outside observer about the object itself\citep{misc:rfid} with active research in making it low-cost, low-power\citep{misc:rfid2}.

The geographical location in the IoT is critical, especially for sensor objects where the location provides important context for the information of which something/someone is accessing\citep{misc:locationMatters}.
For example if there is a station for bikes in a bike sharing system that needs to provide information regarding the amount of bikes at the station, it is critical that it also provides the actual location of the station or the information will not be of any use.
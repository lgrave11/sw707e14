\section{Existing Systems}
A part of analysing and designing a city bike system is to investigate the current solutions on the market. 
Therefore an evaluation of existing systems will be conducted.
We found several existing systems implementing different aspects and features of a city bike system. 
These systems are listed below:
\begin{itemize}
\item Bycyklen in Copenhagen (gobike)
\item Cibi by AFA JCDecaux
\item Alta Bicycle Share
\end{itemize}
\subsection{Bycyklen in Copenhagen}
The bike share system in Copenhagen is called Bycyklen and contains 1,860 bicycles and 100 docking stations. 
Gobike is a Danish/Dutch company that designed this smart system with a bike they call Smart Bike. 
The Smart Bike is equipped with a information screen in the form of a tablet providing the user with an interface to lock the bicycle, select the level of assistance from the electrical system, navigate through the city via GPS and explore new interest points such as cafés and shops.
Furthermore a user of the bicycle is able to check bus or train arrivals near their current location.
The Smart Bike uses the GPS to send the location, who is travelling, and other statistics about the bicycle such as battery life to Gobike Admin.
Gobike is currently seeing possibilities such as location based marketing, and adjusting the traffic lights according to the cyclist based on the pattern of the bike trip and also the weather such as wind speed.
Bycyklen has a very simple 3-step process to rent a bicycle.
\begin{itemize}
\item[Step 1] Book a Smart Bike ahead from your computer or tablet, find a Smart Bike and log in via the on-board tablet.
\item[Step 2] Unlock the by through the tablet, drive around in the city, possibly assisted by the GPS, paying by the hour.
\item[Step 3] Return the Smart Bike, lock the bicycle and log out of the system via the tablet.
\end{itemize}
The Smart Bike enables the user to lock the bicycle and securely park it during the bike trip.

\subsection{Cibi By AFA JCDecaux}

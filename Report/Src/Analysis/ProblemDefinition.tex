\section{Problem Definition}
With the analysis of \bycykel and other similar existing systems performed, it indicates that there is room for improvement.
This leads to the declaration of the problem definition.
It is found that \bycykel is very basic when it comes to technologies, and does not utilise the internet.

In the similar existing systems, GPS, SMS, and card registration is used to locate bicycles.
Furthermore, booking is possible in several of these systems.
This gives inspiration to improvements that can be performed regarding \bycykel, and leads to our hypothesis which is defined as follows.

\begin{center}
\textbf{It is possible to develop a system that makes it easier to use \bycykel, within the context of Internet of Things.}
\end{center}

In order to verify this hypothesis, the following questions have to be answered:

\begin{enumerate}
	\item What are the requirements for a city bicycle booking and positioning system?
	\item How can the booking and positioning system be designed and implemented?
	\item Why should the developed system be used over the currently used system?
\end{enumerate}
\section{Problem Definition}
With the analysis of \bycykel and other similar existing systems performed, it indicates that there is a lot of room for improvement.
It is found that \bycykel is very basic when it comes to technologies.

In other similar existing systems, GPS, SMS, and card registration is used to manage the bicycle experience for the users.
This gives inspiration to improvements that can be performed regarding \bycykelwithoutspace, and leads to our hypothesis which is defined as follows.

\begin{center}
\textbf{It is possible to develop a system that makes it easier to use \bycykelwithoutspace, within the context of Internet of Things.}
\end{center}

In order to verify this hypothesis, the following questions have to be answered:

\begin{enumerate}
	\item What are the requirements for a city bicycle booking and positioning system?\fxwarning{Positioning system? Er det ikke en mere avanceret feature?}
	\item How can the booking and positioning system be designed and implemented?\fxwarning{Og igen}
	\item Why should the developed system be used over the currently used system?
\end{enumerate}
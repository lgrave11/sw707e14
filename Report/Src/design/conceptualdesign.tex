\section{Conceptual Design}
In this section we describe the concept of the overall system that being how a user should be able to use the system, but also the constraints on usage built in to the system.

\subsection{Booking}
A user is able to get an overview of the available bicycles at each station through a website. 
On the website, it is possible for the user to book a bicycle, if a bicycle is available for booking at the station he wants to book. 
In order for a user to perform such a booking, he first needs to create a profile and login, then he has to specify the time and station for his booking.

The reason the user has to specify the time for his booking, originate from the idea that the system has make the bicycle unavailable to others and therefore has to lock the bicycle.
A user is able to book a bicycle up to one day before he actually needs to use it, and in order to avoid that all bicycles are locked, as a consequence of possible long period of lock time for each bicycle, the system only locks a bicycle one hour before the specified time.
This gives a more dynamic system because bicycles remain available for use in the mean time.

Once the user arrives at the station at the specified time, he enters a password generated by the system in order to unlock his bicycle.

To avoid that a bicycle keeps being locked in case a user forgets his booking, the system gives a deadline of one hour after the booking time.
If the bicycle has not been retrieved at the deadline, the system automatically unlocks the bicycle thereby returning the bicycle to the pool of available bicycle at that station. 

%WE NEED TO WRITE THIS ASAP
%1 the idea is that if a booking is not used in a given time frame at the start_time, the booking will be removed, in order to free the bicycle for other to use.
%kig fxwarnings med CONCEPT tekst igennem
\section{Conceptual Design}
In this section we describe how a user should be able to use the system, but also the constraints on usage built in to the system.

\subsection{Booking}
A user is able to get an overview of the available bicycles at each station through a website. 
On the website, it is possible for the user to book a bicycle, if a bicycle is available for booking at the station he wants to book. 
In order for a user to perform such a booking, he first needs to create a profile and login, then he has to specify the time and station for his booking.

The reason the user has to specify the time for his booking, originate from the idea that the system has to make the bicycle unavailable to others and therefore has to lock the bicycle.
A user is able to book a bicycle up to one day before he actually needs to use it, and in order to avoid that all bicycles are locked, as a consequence of possible long period of lock time for each bicycle, the system only locks a bicycle one hour before the specified time.
This gives a more dynamic system because bicycles remain available for use in the mean time.

Once the user arrives at the station at the specified time, he enters a password generated by the system in order to unlock his bicycle, further elaborated in \secref{sec:conceptretrieveandshit}.

To avoid that a bicycle keeps being locked in case a user forgets his booking, the system gives a deadline of one hour after the booking time.
If the bicycle has not been retrieved at the deadline, the system automatically unlocks the bicycle thereby returning the bicycle to the pool of available bicycle at that station. 

\subsection{Retrieving a Bicycle}\label{sec:conceptretrieveandshit}
A user has to be able to retrieve a bicycle at one of the stations located around the city of Aalborg.
Two cases exist, the user may have a booking or he may not have a booking.

If a user has a booking a password is associated with this booking that the user can read on the website.
When the user then walks to the specified station in a time range of one hour before the specified time to one hour after, since that is the time range of the locking, he can enter the password at the station automata and get a message telling him to walk to a given dock to retrieve his bicycle, where he then has to deposit 20DKK to have his bicycle unlocked.
That is, for a bicycle with a booking to be unlocked, it is a two step process of first unlocking it at the automata and then use the regular 20DKK deposit.

If the user does not have a booking he does not have to use the automata, but can instead walk to a free bicycle and deposit 20DKK to retrieve it.
In order to avoid that a user without a booking retrieves a bicycle that has just been unlocked from the automata by another user, we imagine a distinction of reserved unlocked bicycles and free bicycles.
This could be in the form of some status sign at the docks indicating whether a given bicycle is locked, unlocked with booking, or free to use.

\subsection{Deliver a Bicycle}
When a user is finished using his bicycle he has to deliver it back to some nearby station.
In order for him to avoid bicycling to various stations to find one that has free docks, he instead uses the website to locate a nearby station with free docks.
Once such a station is located he bicycles to this station and places his bicycle in a free dock, which ends his usage of the given bicycle and he gets the 20DKK deposit back.

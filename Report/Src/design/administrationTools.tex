%!TEX root = ../../main.tex
\section{Administration Tools}\label{sec:designAdminTools}
In order for the administrator of the bicycle system to make reasonable decision about future improvements towards the bicycle system, statistical data has to be collected and shown to the administrator. This then provides an overview of the usage of the bicycles and prepare him to make an informed decision or provide him with a knowledge of how the system is used.
A list of questions should be able to be answered or hinted by the system, however, the implemented features in the final system is likely limited by available time:

\begin{description}[style=nextline]
\item[Which routes are used?] If routes could be determined, patterns could possibly be seen, e.g. if there is a particular route that is very popular.
\item[Where is the most traffic of bicycles during some period?] An administrator would be able to see how many bicycles leaves and arrives at each station and thereby get an overview of where the traffic is high and low, providing an indication of where to put focus for relocation of bicycles and expansion of stations.
\item[How does the amount of bicycles at a given station change over time?] The administrator can choose a long time interval, providing a general overview of when the activity at each station is high or low, also providing him with an overview of usage at each station.
\item[What is the status of the stations?]
An overview page of the stations should be implemented, providing information about stations such as if they are online or not and what their status is with regards to usage.
\item[What are the hotspots for bicycles?] If positions of bicycles could be determined, it would mean that different kind of patterns could be detected, for example detecting stagnant bicycle hotspots could provide valuable knowledge about where to add new stations.
\end{description}

More details about how this is stored in the database can be seen in \secref{sec:ERdiagram}.
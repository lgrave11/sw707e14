\section{Synchronisation}\label{sec:sync}
As the stations and the website are located on different addresses and communication is through TCP, you have to consider how to synchronise the various parts of the system, in order to avoid an inconsistent state.

Several types of such inconsistent information could happen if you are not careful.

A scenario that could lead to an inconsistent state, if not handled well, is what happens when a station fails to communicate its current state to the global system.
A station maintains a local database holding information about each current booking made at that station.
When a booking has been processed, meaning that the user has taken the bicycle associated with the booking, the state of the local database changes in that the booking does not exist any more.
This change in state is also communicated to the global database maintained by the global system.

A problem can occur when trying to communicate the change as the connection can be interrupted.
This problem is then, formulated with other words, which database holds the correct state of the overall system?
This depends highly on the direction of communication failure. 
In the described case above, the communication failure lies with the station software resulting in the local database of the station having the correct and current state of the system, which is when it comes to the state of the docks.


But if the direction of communication is the opposite, which would be involving bookings registered, the global system would have the correct booking information.

The desired behaviour of the system is to always be in a correct state, which requires that when a communication failure happens, the part participating in the communication holding the correct state resends its information, or the one with the inconsistent state reads the state from the consistent counterpart.

There are two cases for the communication between the station software and the global system.

\begin{description}[style=nextline]
\item[Station to Global System]
A communication failure in this direction results in a faulty state of how many bicycles are available at the given station thereby resulting in a rather useless website for booking.
Additionally, this can also involve the global system having the incorrect amount of docks registered, if such a change were performed at the station.
\item[Global System to Station]
A communication failure in this direction results in a failing booking system, in that no booking made at the website reaches the station and thereby has no effect at all.
In the end this results in a faulty state of how many bicycles are available at the given station because the station does not know of the booking and thereby not lock a bicycle, resulting in the system thinking the booking is registered and the bicycle is locked whereas the station has no knowledge of said booking.
\end{description}


How these cases are tackled is described in \secref{sec:implsync}.
% station --> web : update, insert
% web --> station : insert, delete, update

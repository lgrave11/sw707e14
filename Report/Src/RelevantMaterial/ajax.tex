\section{Asynchronous JavaScript and XML}
Asynchronous JavaScript and XML (AJAX) is a way to create a website update parts of the website without user interaction.
By using AJAX parts of the websites can be updated without reloading the whole website every time, which makes the use of a website much more smooth.
Even though XML is a part of the AJAX name, it is no necessary to use when using AJAX, other methods to send the data can be used, an example could be to use JSON.
When using JSON, data is encoding where after it is printed on a asynchronous website, which the website then decode and reload the part that it effects.

The reason to use AJAX is to improve the usability of a website, as well as the performance of a page.
An example of this is e.g. when rating a film on www.IMDB.com then when pressing the rate, the website does not reload, but register and call the database that an update have occurred. \fxnote{kilde her}
The AJAX should be used whenever the user makes an interaction that does not need to do anything to the website, but only update information on the database, this could be changing the password of a user. 
Furthermore, AJAX should be used when updating only a part of the website, so that the website should not be refreshed before updating this part of the website.
\section{Asynchronous JavaScript and XML}
Asynchronous JavaScript and XML (AJAX) is a way to update parts of a website without user interaction.
By using AJAX, parts of the websites can be updated without reloading the whole website every time, which makes the user experience of a website smoother.
Even though XML is a part of AJAX, it is not necessary to use when using AJAX, other methods to send the data can be used, an example could be to use JSON.
When using JSON, data is encoded and then displayed on a page by itself, which the javascript then decodes and can use to reload the affected parts.

The reason to use AJAX is to improve the usability of a website, as well as the performance of a page.
An example of this is when rating a film on www.imdb.com \citep{misc:imdb} then when a user rates a movie, the website does not reload, but instead calls for an asynchronous update to the database giving the user an non-interrupted experience.
AJAX should be used whenever the user interacts with something that does not need to update the entire page, but only update information on the database, this could be changing the password of a user or it should be used when updating only a part of the website, so that the website should not be refreshed before updating this part of the website.
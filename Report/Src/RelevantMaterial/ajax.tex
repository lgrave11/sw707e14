\section{Asynchronous JavaScript and XML}
\textbf{A}synchronous \textbf{J}avaScript \textbf{a}nd \textbf{X}ML (AJAX) is a way to update parts of a webpage without user interaction.
By using AJAX, parts of the webpages can be updated without reloading the whole webpage every time, which makes the user experience of a website smoother.
Even though XML is a part of AJAX, it is not necessary to use when using AJAX, other methods to send the data can be used, an example could be to use JSON.
For the data transmission, data is obtained per request in some encoding, which the JavaScript then decodes and can use to reload the affected parts.

The reason to use AJAX is to improve the usability of a website, as well as the performance of a page.
An example of this is when rating a film on IMDB \citep{misc:imdb} then when a user rates a movie, the webpage does not reload, but instead utilises an asynchronous update to the database giving the user an non-interrupted experience.
AJAX should be used whenever the user interacts with something that does not need to update the entire page, but only update information on the database, this could be used when updating only a part of the webpage.
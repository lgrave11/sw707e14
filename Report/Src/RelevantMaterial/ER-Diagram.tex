%!TEX root = ../../main.tex
\section{Entity--Relationship Diagram}\label{sec:ERdiagram}
In order to have an idea of how the database handles bookings, location of bicycles, and their relation to the location of the stations, an ER Diagram was made, see \figref{fig:er-dia}.

\begin{figure}
	\centering
	\includegraphics[scale=0.5, trim=0cm 0cm 7cm 0cm, clip]{relevantmaterial/erdiagram}
	\caption{The ER Diagram for database overview.}\label{fig:er-dia}
\end{figure}

As seen in the diagram, the entities consists of \texttt{bicycle}, \texttt{dock}, \texttt{station}, \texttt{booking}, and \texttt{account}.
An \texttt{account} consists of a \texttt{username}, \texttt{email}, which both must be unique, a \texttt{phone} number, \texttt{password} and a \texttt{role} which can be either a regular user or an administrator.
These attributes are common for an \texttt{account} entity, with the \texttt{phone} number being special in that the idea is that they should, in the future, be able to receive the booking password over SMS.

In relation to this is the \texttt{booking} entity, where it can be seen that a user can have many bookings whereas a booking is registered to one and only one account.
The \texttt{booking} then has a \texttt{booking_id}, to uniquely identify a booking, a \texttt{start_time}, such that a station knows when to lock a bicycle, and a password, used to unlock a bicycle if a correct password is entered in a time frame around the \texttt{start_time}.
Recall that the idea is that if a booking is not used in a given time frame at the \texttt{start_time}, the booking will be removed, in order to free the bicycle for other to use.

A \texttt{booking} is also tied to a \texttt{station}, such that a \texttt{booking} is located at one and only one \texttt{station}, whereas a \texttt{station} can have many bookings.

A \texttt{booking} is also tied to a \texttt{bicycle}, such that when the booking is fulfilled it will identify which bicycle was taken.

A \texttt{station} has the attribute \texttt{station_id}, to uniquely identify the station.
Furthermore, it has a \texttt{name}, which is thought to be used to give a meaningful description of a given station.
We are aware that \texttt{name} could be used as a primary key, but having an integer id as the primary key reduce storage requirements when using foreign keys to the station id's.
%unlocks the possibility of duplicate names, which might be desired for stations located at different spots in the city, example could be two stations named "AAU Bycykel Station" which is a quite generic name.

In addition to this, a \texttt{station} has a \texttt{location}, which is used to easily place a station on a map, and can also be used for calculations such as distance between a given station and bicycle.
A \texttt{station} also has two derived attributes, which are \texttt{available_bicycles} and \texttt{amount_of_bookings}, where \texttt{available_bicycles} is beneficial to show to the user, such that they can see if it is possible to gather a bicycle at a given station.

Next is the \texttt{dock} entity, which is a weak entity, as it cannot exist without a \texttt{station}.
This represents the real-life situation with stations located around the city, and each of these has a number of docks.
The dock only has one attribute, which is the \texttt{dock_id}, used to identify a dock in combination with a \texttt{station_id}.

The interesting part of a \texttt{dock} is its relation with a \texttt{bicycle}.
A \texttt{dock} may or may not hold a \texttt{bicycle}, which is represented with the holds relation.

The \texttt{bicycle} entity then consists of a \texttt{bicycle_id}, to uniquely identify a given \texttt{bicycle}, and a \texttt{location}, which can be used to locate lost bicycles.
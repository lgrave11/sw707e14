\section{Simple Object Access Protocol}
The theory of this section primarily based upon \citep{misc:understandingsoapandrest, misc:restsoapwhen}.

The \textbf{S}imple \textbf{O}bject \textbf{A}ccess \textbf{P}rotocol (SOAP) is a protocol for sending structured messages over a computer network using XML \citep{misc:soapintro}.
The use of a SOAP helps structuring message passing, providing type information for the parameters and return value. 

SOAP is useful because at the same time as allowing transmission of information, it also allows implementation of functionality for what is to be done once that information has been received.

A reason to use SOAP over other methods for transmitting information is because it provides more possibilities for security and is generally considered more reliable, and because it provides a standard practice that makes it easier for new developers to pick it up. 

We chose to use SOAP because it provides a standard protocol that when implemented is easy to use for other parts of the system.
Other methods do exist, but SOAP was sufficient for our purposes and is a technology we knew beforehand making it easily adaptable.
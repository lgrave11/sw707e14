\section{The Internet of Things}
This section introduces the Internet of Things (IoT), along with examples of how it is used. The section also introduces various hardware considerations to provide the context in which real-world applications for the IoT are developed.

The IoT is the concept describing the interconnection of uniquely identifiable objects in the problem domain.
The IoT also includes the virtual representation and the interfaces that allow for manipulation and information retrieval regarding objects \citep{misc:InternetOfThingsDefinition, misc:InternetOfThingsDefinition2, misc:InternetOfThingsDefinition3}.

Real usage of the IoT manifests as systems to provide some service, either by informing the user regarding things or allowing an intelligent system to manage those objects.
An example of this could be a so-called `smart home' that allows you to use a single device to manage the connected objects in your home, such as the lights or the oven in the kitchen \citep{misc:InternetOfThingsExamples}.
One society-wide use for it is the idea of a `Smart Grid', where the electricity usage is monitored and managed by an intelligent system that will e.g. redirect electricity if a cable has been cut somewhere in the system \citep{misc:smartGrid}.

There are already a lot of objects in the IoT, and by 2020 it is estimated that there will be upwards of around 26-30 billion things \citep{misc:IoTGrowth1,misc:IoTGrowth2}.
This will likely require a jump to the IPv6 protocol as the amount of IP addresses are severely limited by IPv4 \citep{misc:numberOfAddresses} given that the additions to IPv4, such as multiple devices sharing a single IP address, will at some point become insufficient.

One important aspect of the things connected to the IoT will be what technology they use to connect, here WiFi or mobile networking will not work well as they will likely interfere with one another.
The connectivity technology has to be low-power, cheap, and non-interfering.
One way to do that would be to directly connect objects to a server with a cable connecting to the internet, though that is not practical for objects that must be mobile.
RFID (Radio Frequency Identification) chips can provide relevant information to an outside observer about the object itself \citep{misc:rfid} with active research in making it low-cost and low-power \citep{misc:rfid2}.

The geographical location in the IoT is a good idea, especially for sensors where the location provides important context for the information of which something is accessing \citep{misc:locationMatters}.
For example if there is a station for bicycles in a bicycle sharing system that needs to provide information regarding the amount of bicycles at the station, it is important for the usage of the information that it also provides the actual location of the station, if that is not otherwise known.

In order to give more detail on IoT, aspecs about identification, communication, and software is given.

\subsection{Identification}

\subsection{Communication}

\subsection{Software}



In summary, the IoT has many challenges but at the same time also provides opportunities and benefits for society in general.
There are of course also criticism to it, primarily from a privacy point-of-view because it provides too much information.
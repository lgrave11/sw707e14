To conclude the project, the three questions from the problem definition have to be answered in order to confirm or deny the hypothesis, see \secref{sec:probdef}.

The first question is:
\begin{center}
	\textit{What are the requirements for a city bicycle booking and positioning system?}
\end{center}

The requirements were split into three categories, system criteria, software requirements, and hardware requirements.
For the system criteria an emphasis was put on not redesigning the entire existing system but as an extension instead, as well as an emphasis on the solution being easy to use and access for the target audience.
The emphasis of not redesigning the entire solution has been achieved in the way that a booking system is an additional overlay not necessary to be used, but can be beneficial.
The prioritisation of a simple solution has been an important focus throughout the development of the system, and as is found through the usability tests, the product is generally an easy to use solution.

A list of software requirements were stated, all of which have been fulfilled except the lowest prioritised requirement of predicting the availability of bicycles. The reason for this is that the last requirement is a project on its own, but the system can function fine without it.

The hardware requirements were specified as two required interfaces, interface between station and server, and interface between bicycle and server.
These two interfaces were implemented with help of SOAP and TCP. 
If you were to use the product for actual use, the simulation of hardware would have to be replaced with actual hardware.

The second question is:
\begin{center}
	\textit{How can the booking and positioning system be designed and implemented, within the context of the Internet of Things?}
\end{center}

The product has been designed and implemented in the form of a website, a program to simulate stations, and a program to simulate bicycles travelling around the city of Aalborg.
The website enables the user to perform bookings, though this does not guarantee a bicycle but increases the chance of one, and provides the administrators with a series of graphs indicating the usage of the system as well as enabling the administrators to add and remove the bicycles and stations connected to the system.
The product lies within the context of the Internet of Things in that bicycles, stations, and docks, are the things we keep track of and we use common internet technologies such as SOAP and TCP to do so.
Our product manifests in the form of a service that informs the user about the status of bicycles, station, and docks as well as enabling the users to register bookings to have bicycles locked at stations.

The third question is:
\begin{center}
	\textit{Why should the developed system be used over the currently used system?}
\end{center}

The developed system should be used over the currently used system since it is an extension of the existing system and does not dramatically change any existing features.
That is, the system can be used in a similar fashion to the existing one in that the station software can replace the old lock. 
Additionally, the new system provides for status of the bicycles as well as statistics, which the old system was lacking. 
Furthermore, the booking feature is believed to be a powerful one, but if not desired by Aalborg Kommune, this part could be disabled and still provide useful status information for the user and statistics for the administrators.

Finally the hypothesis to verify is:
\begin{center}
	\textbf{It is possible to develop a system for \bycykel that is user friendly and manageable, within the context of the Internet of Things.}
\end{center}

A system has been developed that makes it possible to perform bookings of bicycles.
The system lies within the context of the Internet of Things as it keeps track of things as well as intercommunication between things that is stations, docks, bicycles, and central server.
This relates to the system providing status information to the users, enabling them to more easily decide which station to go to.
This suggests a user friendly experience and is also supported with the general positive responses from the usability tests.
The system is arguably also more manageable in that statistics for the usage of bicycles exist with the developed system, whereas no statistics were provided beforehand.
On the other hand, more maintenance of the stations would be required as the system has been extended, however, this is a problem setting to be considered by Aalborg Kommune.
 
However, with that in mind the solution generally offers additional features, where Aalborg Kommune can decide which part they would like in their system.
The request of Aalborg Kommune was to be able to monitor the usage of bicycles in Aalborg more, and this feature is also included in the solution in form of the administrator part of the system.
This concludes that the hypothesis can be verified.
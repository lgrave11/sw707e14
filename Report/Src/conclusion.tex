To conclude the project, the three questions from the problem definition have to be answered in order to validate or invalidate the hypothesis, see \secref{sec:probdef}.

The first question was:
\begin{center}
	\textit{What are the requirements for a city bicycle booking and positioning system?}
\end{center}
The requirements were split into three categories, system criteria, software requirements, and hardware requirements.
For the system criteria an emphasis was lain on not redesigning the entire existing system but as an extension instead, as well as an emphasis on the solution being easy to use and access for the users.
The emphasis of not redesigning the entire solution has been achieved in the way that a booking system is an additional overlay not necessary to be used, but can be beneficial.
The prioritisation of simple solution has been existent throughout development, and as is found through the usability tests, the product is generally in a good state in relation to an easy to use solution.

For the software requirements a list of requirements was stated, all of which has been fulfilled except the lowest prioritised requirement of predicting the availability of bicycles. The reason for this is that the last requirement is a project on its own, but the system can function fine without it.

The hardware requirements was specified as two required interfaces, interface between station and server, and interface between bicycle and server.
These two interfaces were implemented with help of SOAP and TCP, but if you were to use the product for actual use, the simulation of hardware would have to be replaced with actual hardware.

\begin{enumerate}
	\item How can the booking and positioning system be designed and implemented?
	\item Why should the developed system be used over the currently used system?
\end{enumerate}


\begin{center}
	\textbf{It is possible to develop a system that makes \bycykel more user friendly and manageable, within the context of Internet of Things.}
\end{center}

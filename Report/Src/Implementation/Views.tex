\section{Views}
This section will describe the view layer and what it does, illustrating by example. For more detail about Model-View-Controller architecture, see \secref{sec:mvc}.

In \lstref{lst:homeIndexView} a truncated version of the home index view can be seen. 
Most of it is normal HTML, but the important part is how it utilizes variables set in the controller, because a view included by the controller gives it access to its variables. 
This allows it to use them to generate more HTML as can be seen lines 6-10 where it uses the array of stations set in the controller to output option elements in a select element.
It can also use defined methods, as shown in line 13-19.

\begin{lstlisting}[language=html, label=lst:homeIndexView, caption={Home Index View}]
[...]
<form action="/Home/Book/" method="post">
    [...]
    <select name="station" id="stations" style="width: 243px;" onchange="UpdateMarker()">
    <option value="0" disabled selected>- Select Station -</option>
    <?php
    foreach($stations as $station){
        echo '<option value="'.$station->station_id.'">'.$station->name.'</option>';
    }
    ?>
    </select><br />
    [...]
    <?php
        if (Tools::isLoggedIn()){
            echo '<input type="submit" value="Book" />';
        } else {
            echo '<a href="/User/Login/">Login</a>';
        }
    ?>
    [...]
</form>
[...]
\end{lstlisting}

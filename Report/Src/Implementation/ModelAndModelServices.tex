%!TEX root = ../../main.tex
\subsection{Model \& Model Services}
This section describe the implementation of the model layer. 
We illustrate this by giving an example model, showing what it looks like and what it does.
A diagram of the model layer can be seen in \appref{app-arch:model}.

The \texttt{Bicycle} entity class can be seen in \lstref{lst:bicycleModel}.
As can be seen, what it does is to capture the attributes of the Bicycle entity from the database.

\begin{minipage}{\textwidth}
\begin{lstlisting}[language=php, label=lst:bicycleModel, caption={Bicycle Class.}]
<?php
class Bicycle
{
    public $bicycle_id = null;
    public $longitude = null;
    public $latitude = null;

    function __construct($bicycle_id, $latitude, $longitude){
        $this->bicycle_id = $bicycle_id;
        $this->longitude = $longitude;
        $this->latitude = $latitude;
    }
}
?>
\end{lstlisting}
\end{minipage}

Every entity have a corresponding service that contain the manipulation and handling of objects of the same type as its corresponding entity. 
Each model service implement an interface enforcing the class to implement methods such as \texttt{validate}, \texttt{create}, \texttt{update}, and \texttt{delete}, for an overview see \figref{fig:websitestructure}. 
Enforcing these methods to exist in every service dictates the responsibility of the service, for example enforcing a \texttt{validate} method dictates that it is the responsibility of the service to ensure a correct format of the model entity before making any changes to the database. 
Furthermore, CRUD dictates that the service should handle all contact with the database concerning its specific entity.

It is also the responsibility of the model to handle all reads from the database, although it is not included in the implemented interface.
This is because it may not always make sense to have a method that can read an entity from the database, and thereby this would possibly never be used.
The reason for this is because what you want to read varies a lot, e.g. do you want to read multiple rows from the database or a single row.
This is in contrast to delete or update that works on a single row.

With the model layer described, we take a look at how the controllers are constructed, and how they use the model layer.
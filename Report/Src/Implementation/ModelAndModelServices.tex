\section{Model \& Model Services}
This section will the model layer, specifically going into what a specific model and what its associated model service does.

The Bicycle model can be seen in \lstref{lst:bicycleModel}. Our modification to the original MVC skeleton here is then that instead of using the models to implement behaviour, we use it to model the corresponding database table.

\begin{minipage}{\textwidth}
\begin{lstlisting}[language=php, label=lst:bicycleModel]
<?php
class Bicycle
{
    public $bicycle_id = null;
    public $longitude = null;
    public $latitude = null;

    function __construct($bicycle_id, $latitude, $longitude){
        $this->bicycle_id = $bicycle_id;
        $this->longitude = $longitude;
        $this->latitude = $latitude;
    }
}

?>
\end{lstlisting}
\end{minipage}

The corresponding BicycleService, though very truncated, can be seen in \lstref{lst:bicycleService}. The BicycleService implements CRUD methods for the Bicycle model.

BicycleService has the following methods,

\begin{itemize}
\item \lstinline|__construct($database)|: Constructs the BicycleService, with the Controller's database instance.
\item \lstinline|create($bicycle)|: Creates a new bicycle in the database.
\item \lstinline|read($id)|: Reads a bicycle from the database based on id.
\item \lstinline|readAll()|: Reads all bicycles from the database.
\item \lstinline|update($bicycle)|: Updates a bicycle in the database with new information.
\item \lstinline|delete($bicycle)|: Delete a bicycle in the database.
\item \lstinline|validate($bicycle)|: Validates a bicycle.
\end{itemize}

\begin{lstlisting}[language=php, label=lst:bicycleService]
<?php

//create read update delete
class BicycleService implements iService
{
    private $db = null;

    function __construct($database){
        try{
            $this->db = $database;
        }
        catch(Exception $ex){
            exit("Unable to connect to database " . $ex);
        }
    }

    /**
     * Function that creates a new bicycle
     * @return the created object
     */
    public function create($bicycle)
    {
        if(validate($bicycle))
        {
            $stmt = $this->db->prepare("INSERT INTO bicycle(longitude, latitude) VALUES (?,?)");
            $stmt->bind_param("dd", $bicycle->longitude, $bicycle->latitude);
            $stmt->execute();
            $id = $this->db->insert_id;
            $stmt->close();
            return new Bicycle($id, null, null);
        }
        else
        {
            return null;
        }
    }

    public function read($id)
    {
        // Read from database.
    }
    
    // [..]

    public function update($bicycle)
    {
        // Update in database.
    }

    public function delete($bicycle)
    {
        // Delete from database.
    }

    public function validate($bicycle)
    {
        // Validate bicycle.
    }

}

?>
\end{lstlisting}

These illustrate the model layer well, and the next section will show how they are used in the context of the controller layer.
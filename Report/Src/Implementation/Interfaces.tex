\section{Interfaces}
As is evident from the chosen solution, some communication between the stations and the database has to be established, to retrieve bookings etc..
This communication is established through a series of interfaces.
These interfaces being from website to station, station to database, and bicycle to database.
Each of these interfaces are described hereafter.

\subsection{Website to Station}
The website has to establish contact to a station when a booking or un-booking has been performed.
It is evident that this is important, as the stations need to be notified to keep track of bookings involving themselves as station.

How this contact is established is through TCP.
The call to the notification methods is added in the \textit{BookingService}, as each time a booking is created, updated, or deleted, it is called through BookingService, and as of such we ensure the notification method is always called.

\begin{minipage}{\textwidth}
\begin{lstlisting}[caption = {Function for notifying a station of a new booking.}, label = {lst:notifystationbooking}]
public static function notifyStationBooking($station_id, $booking_id, $start_time, $password)
{
	$message = self::makeJson("booking", $station_id, $booking_id, $start_time, $password);
	
	$sock = socket_create(AF_INET, SOCK_STREAM, SOL_TCP);b
	$sock_data = socket_connect($sock, self::$station_ips[$station_id - 1], self::$port);
	
	$sock_data = socket_write($sock, $message);
	socket_close($sock);
}
\end{lstlisting}
\end{minipage}

In order to see how the notification to a station is performed, see \lstref{lst:notifystationbooking}.
As can be seen in the code, on line 3, a message is JSON encoded with an action, station-id, booking-id, start-time, and password.
This is performed to have a standard way of encoded data, which can be easily decoded by the station.
Afterwards in line 5-9, the tcp-socket is constructed and sent to the affected station.

It is then up to the station to read this information and handle it in the station's local database.

As can be seen from the code, the action is in this case "booking", but to notify with an un-booking this difference is that the action is "unbooking" and you only need the station- and booking-id.

\subsection{Station to Database}
The station to database interface is important to register whether bicycles has been taken at stations, in order to give correct information on the website status page.
Additionally the interface is used to retrieve information about bookings registered in the large database, in case of power-up of a given station.

The interface has been implemented in a SOAP encoding by help of the NuSOAP library.\fxwarning{kilde her}
This library makes you able to write regular PHP functions and then register these with the NuSOAP library, in order to have a webservice generated.
Bare in mind that all methods registered with NuSOAP needs to have a return value, and could not figure out how to specify void, as of such, a dummy boolean value is used where no other return value is needed.

An example of the implementation of an update and read operation on the database is presented hereafter.

\begin{minipage}{\textwidth}
\begin{lstlisting}[caption = {Method for registering a bicycle as been returned to a dock at a given station.}, label = {lst:bicycledockstationreturned}]
$server->register('BicycleReturnedToDockAtStation',
	array('bicycle_id' => 'xsd:int',
	'station_id' => 'xsd:int',
	'dock_id'    => 'xsd:int'),
	array('return' => 'xsd:boolean'),
	$SERVICE_NAMESPACE,
	$SERVICE_NAMESPACE . '#soapaction',
	'rpc',
	'literal',
	'Registers that a given bicycle has arrived at a given dock at a given station'
);
function BicycleReturnedToDockAtStation($bicycle_id, $station_id, $dock_id)
{
	global $db;
	$stmt = $db->prepare("UPDATE dock SET holds_bicycle = ? WHERE station_id = ? AND dock_id = ?");
	$stmt->bind_param("iii", $bicycle_id, $station_id, $dock_id);
	$stmt->execute();
	$stmt->close();
	
	return true;
}
\end{lstlisting}
\end{minipage}

If you take a look at \lstref{lst:bicycledockstationreturned}, the code is split into two parts, line 1-11 and line 12-21, which handles different parts of providing a method to register that a bicycle has returned to a dock at given station.

Line 1-11 takes care of registering the PHP function, such that it can be included in the auto-generation of the SOAP encoded webservice.
As can be seen from this specification, we tell the NuSOAP library the name of the function to register on line 1.
Then line 2-4 specifies the input parameters, line 5 specifies the return value, and line 6-11 is less interesting parts with deals with how the method should be represented in SOAP.

Line 12-21 is the actual method for registering that a bicycle has been return to a dock at a given station.
This is performed with use of prepared statements, as can be seen in line 15-18, as is done to prevent MySQL injection attempts.
As you can see, the actual statement expresses an update on the dock, such that the dock where the bicycle has been placed at gets this information updated.

\begin{minipage}{\textwidth}
\begin{lstlisting}[caption = {Method for reading all bookings for a given station}, label = {lst:getallbookingstation}]
$server->register(
	'GetAllBookingsForStation',
	array('station_id' => 'xsd:int'),
	array('return' => 'tns:BookingObjectArray'),
	$SERVICE_NAMESPACE,
	$SERVICE_NAMESPACE . '#soapaction',
	'rpc',
	'encoded',
	'Get all bookings for station'
);
//in case you want to read everything.
function GetAllBookingsForStation($station_id)
{
	global $db;
	$stmt = $db->prepare("SELECT booking_id, start_time, start_station, password, for_user FROM booking WHERE start_station = ? AND password IS NOT NULL");
	$stmt->bind_param("i", $station_id);
	$stmt->execute();
	$stmt->bind_result($booking_id, $start_time, $start_station, $password, $for_user);
	
	$returnarray = array();
	while($stmt->fetch())
	{
	$to_add_class = new stdclass();
	$to_add_class->booking_id = $booking_id;
	$to_add_class->start_time = $start_time;
	$to_add_class->start_station = $start_station;
	$to_add_class->password = $password;
	$to_add_class->for_user = $for_user;
	
	$returnarray[] = json_encode($to_add_class);
	}
	$stmt->close();
	return $returnarray;
}
\end{lstlisting}
\end{minipage}

An example of a method that reads from the database is the \textit{GetAllBookingsForStation} function, seen in \lstref{lst:getallbookingstation}.
Such a method is for example useful when first booting a station where its local database is empty.

Taking a look at line 1-10, you see the registration of the PHP function for integration in the SOAP specification.
As can be seen, it utilised a custom type called \textit{BookingObjectArray}, which is an array type used to contain multiple string JSON encoded that represents bookings.
If you then take a look at line 15-16, you see how the fields in the booking table is selected.
The results you get from the select statement is then bound to variables as seen in line 18.
Finally on line 21-31 you can see how each booking is JSON encoded and appended to a return array, which is then returned on line 33, and can be used by station to traverse the array returned and decode each string to get its corresponding booking information.

\subsection{Bicycle to Database}
The interface from bicycle to database, is an interface that is needed to register the location of a bicycle if GPS tracking is to be implemented.
The idea is that each bicycle would use the interface to inform the system where the given bicycle is located, according to GPS.

The interface is implemented in the same fashion as the interface from station to database.
As of such, it is implemented as a SOAP web-service, using the NuSOAP library to gain the desired encoding.
There exists one method, the RegisterGPS method, which takes three arguments, the bicycle-id, latitude and longitude.
It is then coded similar to \lstref{lst:bicycledockstationreturned}, but where a given bicycles latitude and longitude is updated instead.
